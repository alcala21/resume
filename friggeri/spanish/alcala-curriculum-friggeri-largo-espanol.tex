%%%%%%%%%%%%%%%%%%%%%%%%%%%%%%%%%%%%%%%%%
% Friggeri Resume/CV
% XeLaTeX Template
% Version 1.2 (3/5/15)
%
% This template has been downloaded from:
% http://www.LaTeXTemplates.com
%
% Original author:
% Adrien Friggeri (adrien@friggeri.net)
% https://github.com/afriggeri/CV
%
% License:
% CC BY-NC-SA 3.0 (http://creativecommons.org/licenses/by-nc-sa/3.0/)
%
% Important notes:
% This template needs to be compiled with XeLaTeX and the bibliography, if used,
% needs to be compiled with biber rather than bibtex.
%
%%%%%%%%%%%%%%%%%%%%%%%%%%%%%%%%%%%%%%%%%

\documentclass[]{friggeri-cv} % Add 'print' as an option into the square bracket to remove colors from this template for printing

\addbibresource{alcala-bibliography.bib} % Specify the bibliography file to include publications
\usepackage[spanish]{babel}

\RequirePackage{marvosym}
\hypersetup{
	hidelinks=true
}

\begin{document}

\header{Carlos Felipe}{Alcalá Pérez}{} % Your name and current job title/field

%----------------------------------------------------------------------------------------
%	SIDEBAR SECTION
%----------------------------------------------------------------------------------------

\begin{aside} % In the aside, each new line forces a line break
\section{contacto}
\Letter~ 507 E. Michigan St, B7F4
Milwaukee, Wisconsin 53202
USA
~
\includegraphics[width = 8pt]{google-icon.png}~ +1 (213) 863 5845
\Mobilefone~ +1 (414) 426 6814
\Telefon~ +1 (414) 524 5886
~
\href{mailto:alcala21@gmail.com}{alcala21@gmail.com}
% \href{http://www.smith.com}{http://www.smith.com}
% \href{http://facebook.com/johnsmith}{fb://jsmith}
\section{idiomas}
Español (nativo)
Inglés (fluido)
\section{lenguages de programación}
{\color{red} $\varheartsuit$} R
Matlab, Python,
\LaTeX, RMarkdown,
VBA \& C\#
\end{aside}

%----------------------------------------------------------------------------------------
%	EDUCATION SECTION
%----------------------------------------------------------------------------------------

\section{educación}

\begin{entrylist}

%------------------------------------------------

\entry
{2007--2011}
{Doctorado en Filosofía {\normalfont en Ingeniería Química}}
{\flushright \href{http://chems.usc.edu}{Universidad del Sur de California.} Los Angeles, California.}
% {Dissertation about fault detection and diagnosis.}

%------------------------------------------------

\entry
{2005--2007}
{Maestría en Ciencias {\normalfont en Ingeniería Química}}
{\flushright \href{http://www.che.utexas.edu/}{Universidad de Texas en Austin.}
Austin, Texas.}
% {\emph{Money Is The Root Of All Evil -- Or Is It?} \\ This thesis explored the idea that money has been the cause of untold anguish and suffering in the world. I found that it has, in fact, not.}

%------------------------------------------------

\entry
{1999--2004}
{Licenciatura {\normalfont en Ingeniería Química}}
{\flushright \href{http://www.itcm.edu.mx/}{Instituto Tecnológico de Ciudad Madero}.
Ciudad Madero, México}

\entry
{1996--1999}
{Bachillerato Tecnológico {\normalfont en Química Industrial}}
{\flushright {Centro de Estudios Tecnológicos Industrial y de Servicios No. 78}.
Altamira, México}

%------------------------------------------------

\end{entrylist}


%----------------------------------------------------------------------------------------
%	RESEARCH INTERESTS
%----------------------------------------------------------------------------------------
% \section{research interests}
% Multivariate Statistical Fault Detection and Diagnosis, Data Analysis, Machine Learning, Tuning and Monitoring of PID Controllers
%----------------------------------------------------------------------------------------
%	WORK EXPERIENCE SECTION
%----------------------------------------------------------------------------------------

\section{experiencia laboral}

\subsection{tiempo completo}

\begin{entrylist}

%------------------------------------------------

\entry
{2011 -- Presente}
{\href{http://www.johnsoncontrols.com}{Johnson Controls, Inc}}
{Milwaukee, Wisconsin.}
{
\emph{Ingeniero de Investigación Nivel Senior}
\begin{itemize}
	\item Desarrollo métodos estadísticos para la detección y diagnóstico de fallas en equipos de aire acondicionado.
	\item Desarrollo métodos para el ajuste automático del sampleo en controladores PID.
	\item Desarrollo métodos para el monitoreo del rendimiento de controladores PID.
	\item Desarrollo métodos estadístico para la detección de estados estacionarios en procesos termodinámicos.
	\item Desarrollo métodos estadísticos para construir de manera recursiva y adaptable modelos basados en análisis de componentes principales (PCA).
	\item Desarrollo de métodos para desacoplar la interacción entre controladores PI.
\end{itemize}
}
%------------------------------------------------

\end{entrylist}

\newpage

\subsection{prácticas}

\begin{entrylist}

\entry
{2010}
{\href{http://www.dow.com/}{The Dow Chemical Company}}
{Freeport, Texas}
{\emph{Practicante de Investigación en el Verano} \\
Desarrollé una aplicación en Excel para el monitoreo de procesos contínuos y en batch usando métodos estadísticos multivariables.}

%------------------------------------------------

\entry
{\begin{minipage}[t]{2cm}2009 \\ 2007 \\ 2006\end{minipage}}
{\href{http://www.capstonetechnology.com}{Capstone Technology}}
{Seattle, Washington}
{\emph{Practicante de Ingeniería en el Verano}
\begin{itemize}
	\item Desarrollé una aplicación para el monitoreo de la eficiencia de combustión en hornos usando análisis estadístico multivariable de imágenes.
	\item Desarrollé una aplicación para el modelado estadístico de procesos químicos usando mínimos cuadrados parciales.
	\item Desarrollé una aplicación para la detección y el diagnóstico de fallas en sensores y procesos químicos usando el método estadístico de análisis de componentes principales.
\end{itemize}}

%------------------------------------------------

\entry
{2008}
{NMC North Microelectronics}
{Beijing, China}
{\emph{Practicante de Ingeniería en el Verano}
\begin{itemize}
	\item Desarrollé una aplicación para el monitoreo de la operación de un proceso de manufacturación de semiconductores usando el método estadístico de análisis de componentes principales.
\end{itemize}}

%------------------------------------------------




\end{entrylist}

%----------------------------------------------------------------------------------------
%	PATENTS SECTION
%----------------------------------------------------------------------------------------

\section{patentes y publicaciones}

\printbibsection{misc}{patentes} % Print all miscellaneous entries from the bibliography

\printbibsection{article}{artículos en revistas indexadas} % Print all articles from the bibliography

\printbibsection{inproceedings}{artículos en congresos} % Print all books from the bibliography

%----------------------------------------------------------------------------------------
%	AWARDS SECTION
%----------------------------------------------------------------------------------------

\section{premios}

\subsection{profesional}

\begin{entrylist}

%------------------------------------------------

\entry
{2015}
{1er Lugar en el TechChallenge de Building Efficiency 2015}
{Johnson Controls}
{El TechChallenge de Building Efficiency es una competencia anual de innovación en la compañía Johnson Controls. Los proyectos finalistas se exponen ante una audiencia y se emiten votos para elegir al ganador.}

\end{entrylist}

\subsection{doctorado}


\begin{entrylist}

\entry
{2007}
{\href{http://www.robertorocca.org/en/fellowships/fellows07.aspx}{Beca Roberto Rocca}}
{}
{Beca que otorga la Fundación Roberto Rocca para estudiantes de Doctorado.}
\end{entrylist}

\newpage

\subsection{maestría}

\begin{entrylist}

\entry
{2006}
{Beca David J. Bruton}
{}
{Beca que otorga la Fundación David J. Bruton para estudiantes de postgrado en la Facultad de Ingeniería de la Universidad de Texas.}

\entry
{2005}
{Beca de la Facultad de Ingeniería de UT Austin}
{}
{Beca de la Faculta de Ingeniería de la Universidad de Texas a estudiantes de postgrado.}

\entry
{2005}
{Beca Fulbright}
{}
{Beca que otorga el gobierno de los Estados Unidos para que estudiantes extranjeros hagan estudios de postgrado en ese país. Es una de las becas más prestigiosas.}


\end{entrylist}

\subsection{licenciatura}

\begin{entrylist}

\entry
{2004}
{Mejor promedio en Ingeniería Química}
{Ciudad Madero, Tamaulipas}
{Instituto Tecnológico de Ciudad Madero}

\entry
{2003}
{1er Lugar en Química}
{Los Mochis, Sinaloa}
{Concurso Nacional de Ciencias Básicas de los Institutos Tecnológicos}

\entry
{2003}
{1er Lugar Global en Física, Química y Matemáticas}
{Matehuala, San Luis Potosí}
{Concurso Regional de Ciencias Básicas de los Institutos Tecnológicos}


\entry
{2002}
{1er Lugar Global en Física, Química y Matemáticas}
{Jiquilpan, Michoacán}
{Concurso Nacional de Ciencias Básicas de los Institutos Tecnológicos}

\entry
{2002}
{1er Lugar Global en Física, Química y Matemáticas}
{Piedras Negras, Coahuila}
{Concurso Regional de Ciencias Básicas de los Institutos Tecnológicos}

\entry
{2001}
{1er Lugar en Física}
{Linares, Nuevo León}
{Concurso Regional de Ciencias Básicas de los Institutos Tecnológicos}

\entry
{2000}
{1er Lugar en Física}
{Saltillo, Coahuila}
{Concurso Regional de Ciencias Básicas de los Institutos Tecnológicos}

%------------------------------------------------

\end{entrylist}

\subsection{bachillerato}

\begin{entrylist}

	\entry
	{1999}
	{Mejor promedio de generación}
	{Altamira, Tamaulipas}
	{Centro de Estudios Tecnológicos Industrial y de Servicios No. 78}

	\entry
	{1999}
	{2do Lugar en Física}
	{Ciudad de México}
	{Concurso Nacional de Ciencias de la DGETI.}

	\entry
	{1999}
	{1er Lugar en Física}
	{Tamaulipas}
	{Concurso Estatal de Ciencias de la DGETI}

	\entry
	{1998}
	{1er Lugar en Física}
	{Durango}
	{Olimpíada Nacional de Física}

	\entry
	{1998}
	{Abanderado del Equipo Mexicano de Física}
	{Mérida, Venezuela}
	{Olimpíada Iberoamericana de Física}

	\entry
	{1998}
	{1er Lugar en Física}
	{Tamaulipas}
	{Concurso Estatal de Ciencias de la DGETI}

	\entry
	{1997}
	{8vo Lugar en Física}
	{Puebla}
	{Olimpíada Nacional de Física}

	\entry
	{1997}
	{3er Lugar en Física}
	{Tamaulipas}
	{Olimpíada Estatal de Física}


\end{entrylist}


%----------------------------------------------------------------------------------------
%	INTERESTS SECTION
%----------------------------------------------------------------------------------------

% \section{interests}

% \textbf{professional:} data analysis, company profiling, risk analysis, economics, web design, web app creation, software design, marketing \textbf{personal:} piano, chess, cooking, dancing, running

%----------------------------------------------------------------------------------------
%	PUBLICATIONS SECTION
%----------------------------------------------------------------------------------------


% \begin{refsection} % This is a custom heading for those references marked as "inproceedings" but not containing "keyword=france"
% \nocite{*}
% \printbibliography[sorting=chronological, type=inproceedings, title={international peer-reviewed conferences/proceedings}, notkeyword={france}, heading=bibheading]
% \end{refsection}

% \begin{refsection} % This is a custom heading for those references marked as "inproceedings" and containing "keyword=france"
% \nocite{*}
% \printbibliography[sorting=chronological, type=inproceedings, title={local peer-reviewed conferences/proceedings}, keyword={france}, heading=bibheading]
% \end{refsection}



% \printbibsection{report}{research reports} % Print all research reports from the bibliography

%----------------------------------------------------------------------------------------

\end{document}